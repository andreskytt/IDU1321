\documentclass[nobib]{tufte-handout}
\usepackage[utf8]{inputenc}
\usepackage[estonian]{babel}
\usepackage[style=english]{csquotes}

\title{IDU1321 - Ettevõtte äriarhitektuur. Iseseisev töö}
\author[Andres Kütt]{Andres Kütt}

%\date{04.08.2017}
%\geometry{showframe} % debug

\usepackage{natbib}

\usepackage{graphicx}
 \setkeys{Gin}{width=\linewidth,totalheight=\textheight,keepaspectratio}
  \graphicspath{{graphics/}} % set of paths to search for images
\usepackage{amsmath}  % extended mathematics
\usepackage{booktabs} % book-quality tables
\usepackage{units}    % non-stacked fractions and better unit spacing
\usepackage{multicol} % multiple column layout facilities
\usepackage{lipsum}   % filler text
\usepackage{fancyvrb} % extended verbatim environments
  \fvset{fontsize=\normalsize}% default font size for fancy-verbatim environments

% Standardize command font styles and environments
\newcommand{\doccmd}[1]{\texttt{\textbackslash#1}}% command name -- adds backslash automatically
\newcommand{\docopt}[1]{\ensuremath{\langle}\textrm{\textit{#1}}\ensuremath{\rangle}}% optional command argument
\newcommand{\docarg}[1]{\textrm{\textit{#1}}}% (required) command argument
\newcommand{\docenv}[1]{\textsf{#1}}% environment name
\newcommand{\docpkg}[1]{\texttt{#1}}% package name
\newcommand{\doccls}[1]{\texttt{#1}}% document class name
\newcommand{\docclsopt}[1]{\texttt{#1}}% document class option name
\newenvironment{docspec}{\begin{quote}\noindent}{\end{quote}}% command specification environment

\begin{document}
\maketitle
\begin{abstract}
\noindent
Siin kirjeldan ainue IDU1321 iseseisva ülesande sisu ja eesmärki
\end{abstract}

\section{Versioonid}
\begin{description}
	\item[1.1] Täpsustatud jooniste formaati ja hindamist
	\item[2.0] Ühtlustatud kruntide/süsteemide terminoloogiat, lisatud tüüpvigade sektsioon
\end{description}

\section{Eesmärk}
Iseiseisva töö eesmärgiks on modelleerida iseseisvalt üks kitsas lõik keerulisest ettevõttest

\section{Metoodika}
\begin{itemize}
	\item Tööd võib teha ka grupiti, kuid ülesanne on mõeldud ühele inimesele. Mida rohkem inimesi on tööga tegelenud, seda rangemalt hindan\sidenote{Jämedalt kehtib valem $H=H_t\times-1.1^{N-1}$ kus $H_t$ on töö puhktisumma, H protokolli minev number ja $N$ on meeskonna suurus. Mida rohkem inimesi, seda paremini peab töö sama hinde saamiseks tehtud olema}
	\item Allpool kirjeldatav struktuur on oluline, konkreetse sisu puhul on määrav arusaadavus\sidenote{Tuletage meelde loengus räägitut: ma panen hinde oma arusaamise ja mitte teie peas toimuva järgi}
	\item Diagrammid võib teha mõne tarkvarapaketiga\sidenote{\url{http://www.sparxsystems.com/} võimaldab tasuta versiooni, \url{https://www.archimatetool.com/} on samuti tasuta} aga piisab ka paberile tehtud joonise fotost\sidenote{Sobib ka näiteks pilt valgele tahvlile tehtud joonisest. Peamine ülesanne on keskmise intelligentsiga lugejale (sh. õppejõud) oma mõtte kohale viimine. Täpne notatsioon ja joonise formaat on olulised vaid niivõrd, kuivõrd nad aitavad kaasa joonisest aru saamisele}
	\item Tulemus peaks olema nii lühike kui võimalik aga mitte lühem
	\item Tulemus tarnitakse PDFina, sest erinevate dokumendivormingute suutlikkus diagramme vussi keerata on liiga suur
\end{itemize}

\section{Hindamine}
Töö hinne moodustab aine lõpphindest \textbf{60\%}. Tööd hinnatakse 100 punkti skaalal,  järgmistest kriteeriumitest igaüks on korrektselt täidetuna väärt 20 punkti.
\begin{enumerate}
	\item Joonised on loetavad ja usutavalt mõistetavad tavalise it-inimese poolt
	\item Tekst on loetav, arusaadav ja konsistente\sidenote{Läbivalt stabiilne terminoloogia, puuduvad olulised vasturääkivused, korrektne keelekasutus, jne.}
	\item Kirjeldatud mudel on terviklik\sidenote{Kõik kihid on kirjeldatud, kõigi kihitde kõik elemendid toetuvad alumistele}
	\item Süsteemi skoop on selge ning sellest peetakse kinni\sidenote{Elemendid on kirjeldatud võrdse detailsusega, süsteemist väljapoole jäävad elemendid on markeeritud}
	\item Dokumenti lugedes tekib (usutavalt) tavalisel infotehnoloogiaga igapäevaselt tegeleval inimesel organisatsiooni arhitektuuri konkreetsest lõigust selge ülevaade
\end{enumerate}

\section{Ülesanne}
Te olete ühe väikeriigi maksuameit ettevõttearhitekt. Loomulikult on teil olemas kõrge taseme ülevaade kogu organisatsioonis toimuvast, kuid (kuna tegu on kriitilise äriprotsessiga) on teil tarvis koostada eraisiku tulu deklareerimise organisatsiooni arhitektuur. Eesmärgiks on luua \textbf{dokument, mis võimaldab jälgida kõikide osakondade vastutust tehnilise komponendini ning siduda kõik tehnilised komponendid konkreetsete kasutuslugude, sealtkaudu äriprotsessi sammude ja edasi organisatsiooniliste üksustega.}
\begin{marginfigure}%
  \includegraphics[width=\linewidth]{bpmn.png}
  \caption{Näide seoste kujutamisest eri kihtide vahel}
  \label{fig:bpmn}
\end{marginfigure}


Teie töö vastu tunnevad huvi järgmised osakonnad
\begin{description}
	\item[Teenuste osakond] Vastutab lõppkasutaja kogemuse ja interakstsiooni eest, nende hallata on ka deklaratsiooni vorm ning sellele kehtivad (ranged ja keerulised) reeglid
	\item[Kontrolli osakond] Vastutab kontrolli eest, mille kõik esitatud tuludeklaratsioonid läbivad
	\item[Arvelduste osakond] Hoolitseb, et toimuks riigi tulude arvestus ning et kodanikud oma enam-makstud tulumaksu pärast kontrolli tagasi saaksid
\end{description}

\section{Tulem}
\begin{marginfigure}%
  \includegraphics[width=\linewidth]{kasutuslood.png}
  \caption{Eri kihtide objektid samal diagrammil}
  \label{fig:uc}
\end{marginfigure}

Tulemis on esindatud järgmised kihid
\begin{description}
	\item[Organisatsioon] Mis organisatsiooni osad protsessis osalevad\sidenote{Vt. eelmist punkti minimaalse hulga osas, alati võib uusi üksusi lisada}
	\item[Äriprotsess] Millistest sammudest koosneb äriprotsess\sidenote{Piisab \enquote{happy day} stsenaariumist ja paarist alternatiivsest harust} ja kuidas on need sammud seotud organisatsiooniga
	\item[Kasutuslood] Mis kasutuslood\sidenote{Võib keskenduda peamisele väärtusprotsessile ja ignoreerida lisategevusi nagu kasutaja tuvastamine} realiseerivad millised protsessi sammud
	\item[Krundid] Millised krundid\sidenote{Linnaplaneerimise metoodikas kutsutakse süsteemi funktsionaalseid osi \enquote{kruntideks}. Neist võib mõelda kui teatavas mõttes allsüsteemidest} leiduvad, millistesse tsoonidesse nad jagunevad ning mis kasutuslood puudutavad milliseid krunte?
	\item[Tehnilised komponendid] Millistest suurematest\sidenote{Klassidiagrammi tasemele ei ole vaja minna aga veebirakendus ja andmebaas võiksid olla markeeritud} komponentidest krundid koosnevad
\end{description}
\begin{marginfigure}%
  \includegraphics[width=\linewidth]{tsoonid.png}
  \caption{Eri kihtide objektid samal diagrammil}
  \label{fig:zone}
\end{marginfigure}

Eritüübiliste objektide vahelise seoste demonstreerimiseks võib kasutada millist iganes selge semantikaga vahendit. Peamised võimalused on
\begin{itemize}
	\item Seosed objektide vahel, nagu näidatud joonisel \ref{fig:uc}. 
	\item Värvid, nagu näidatud joonisel \ref{fig:bpmn}. Eri osakondade sammud on eri värvi. Sobilik, kui diagramm on keeruline ja seoste näitamine läheks liiale
	\item Sisalduvus paketis, nagu näidatud joonisel \ref{fig:zone}
	\item Tabel, kus ridades ja veergudes on seostatavad osised ja seos on lahtrisse märgitud ristiga
\end{itemize}

Muidugi võib kasutada ka eri variantide kombinatsiooni, nagu joonisel \ref{fig:zone}


\section{Tüüpvigu}
Eelmiste aastate kodutööde hindamisel on tulnud välja hulk tüüpvigu, mida tasub vältida
\begin{itemize}
	\item Seosed tsoonide ja mitte kruntidega. Näiteks seos tehnilise komponendi ja kanalitsooni vahel ei ütle lugejale midagi selle kohta, mida too tehniline komponent teeb andes vaid info, et tõenäoliselt paistab komponent kuidagi välisvõrku. Seos otse krundiga annab aga täpse teadmise, millist funktsionaalsust antud tehniline komponent realiseerib. Sama kehtib ka kasutuslugude kohta: seos tsooniga ei anna olulist infot, seos krundiga annab
	\item Seoste puudumine kihtide vahel. Kogu harjutuse mõte on siduda organisatsiooni eri aspekte tervikuks. Seega on seoste puudumine (tüüpiliselt kasutuslugude ja äriprotsessi sammude vahel aga ka mujal) oluliseks puuduseks
	\item Tehniliste komponentide mõiste segamine krundi mõistega. Tehniline komponent on miski, mis realiseerib mingi krundi mingit funktsionaalsust. Seega on krunt nimega \enquote{andmebaas} sisuliselt vale. Andmebaas kui selline realiseerib reeglina mingit funktsionaalsust aga millist?
	\item Tehniliste komponentide liigne abstraktsioon \enquote{serveri} ja \enquote{andmebaasi} tasemele. Jah, me saame teada, et krunti realiseerib andmebaas ja server aga milline? Kui joonisel on toodud üks server ja üks andmebaas, võib sellest järeldada, et küllap ongi kogu funktsionaalsus koondunud ühte tehnilisse komponentide ja ühte andmebaasi aga kui servereid ja andmebaase on mitu, jääb tehniliste komponentide kihi infosisaldus väga lahjaks
\end{itemize}

\bibliography{idu1321}
\bibliographystyle{plainnat}
\end{document}